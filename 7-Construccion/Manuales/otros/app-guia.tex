Este manual tiene como objetivo proporcionarte una guía detallada sobre cómo utilizar todas las características y funcionalidades como guía en nuestra aplicación móvil.

\subsubsection{Requisitos del Sistema}
Antes de comenzar, asegúrate de que tu dispositivo cumple con los siguientes requisitos:

\begin{itemize}
	\item Sistema operativo móvil actualizado (recomendamos las últimas versiones de Android o iOS).
	\item Conexión a Internet estable.
\end{itemize}

\subsubsection{Acceso a la Aplicación}
Para acceder a nuestra aplicación móvil, sigue estos pasos:

\begin{enumerate}
	\item Abre la aplicación descargada en tu dispositivo.
	\item En la pantalla de inicio, puedes explorar las diferentes funcionalidades a través de la barra de navegación en la parte inferior.
\end{enumerate}

\subsubsection{Funcionalidades Principales}
A continuación, describiremos las funcionalidades principales de nuestra aplicación móvil:

\subsubsection{Registro de Usuario}
Para registrarte en nuestra aplicación, sigue estos pasos:

\begin{enumerate}
	\item Toca la pestaña “Cuenta” en la barra de navegación inferior.
	\item Se abrirá una pantalla con opciones para iniciar sesión o registrarte.
	\item Toca el enlace “Registrarse” para acceder al formulario de registro.
	\item Completa el formulario de registro con tu información personal.
	\item Toca el botón “Registrarse” para enviar el formulario.
\end{enumerate}

Al completar el registro con éxito, accederás automáticamente a tu cuenta y podrás comenzar a reservar actividades.

\subsubsection{Inicio de Sesión}
Si ya tienes una cuenta, puedes iniciar sesión siguiendo estos pasos:

\begin{enumerate}
	\item Toca la pestaña “Cuenta” en la barra de navegación inferior.
	\item Ingresa tu email y contraseña en los campos correspondientes.
	\item Toca el botón “Iniciar Sesión” para acceder a tu cuenta.
\end{enumerate}

Al iniciar sesión con éxito, accederás a tu cuenta y podrás comenzar a reservar actividades.

\subsubsection{Exploración de Actividades}
Desde la pantalla de inicio, puedes explorar las actividades pulsando la el botón “Empezar a buscar” de la pagina de inicio. Te llevará a una página donde encontrarás una lista de actividades disponibles, con la opción de filtrar por nombre, por rango de fechas, precio máximo, número de personas, puntuación mínima y idioma.

\begin{itemize}
	\item Para filtrar por nombre, ingresa el nombre de la actividad en la barra de búsqueda superior. A los 3 segundos de haber dejado de escribir, se mostrarán las actividades que coincidan con el nombre ingresado.
	\item Para aplicar filtros adicionales, toca el botón “Añadir filtros” que se encuentra en la parte inferior de la pantalla.
	\item Para filtrar por rango de fechas, selecciona las fechas de inicio y fin en los campos correspondientes.
	\item Para filtrar por precio máximo, ajusta el control deslizante.
	\item Para filtrar por número de personas, utiliza los botones de incremento y decremento.
	\item Para filtrar por puntuación mínima, utiliza los botones de incremento y decremento.
	\item Para filtrar por idioma, selecciona los idiomas deseados tocando las casillas correspondientes.
\end{itemize}

Si deseas modificar o eliminar los filtros, utiliza los botones “Aplicar filtros” o “Eliminar filtros” según sea necesario.

\subsubsection{Ver Información Detallada de una Actividad}
Para ver la información detallada de una actividad, toca el botón “Ver más” de la actividad deseada. Se abrirá una nueva pantalla con la información detallada de la actividad, incluyendo el nombre, la descripción, la duración, la puntuación y la ubicación.

Para ver los precios, idiomas y horarios de la actividad, toca el botón “Ver disponibilidad” que se encuentra en la parte inferior de la pantalla.

Desde esta pantalla podrás seleccionar la fecha en el calendario y el número de personas usando los botones de incremento y decremento. Una vez seleccionada la fecha y el número de personas, podrás ver los precios, idiomas y horarios disponibles para la actividad.

\subsubsection{Ver eventos proximos}
Para ver los eventos que tendrás próximamente, deberás pulsar el icono de menú en la parte superior izquierda de la página de inicio y seleccionar la opción “Eventos próximos” .
Se abrirá una página donde podrás seleccionar la fecha deseada para ver los eventos programados para ese día.
Debajo de esto se mostrará una lista con los eventos programados para la fecha seleccionada, incluyendo el nombre, la descripción, la duración, la ubicación, el horario y los participantes de cada evento.


\subsubsection{Cambiar Datos Personales}
Para cambiar tus datos personales desde tu dispositivo móvil, una vez hayas iniciado sesión, sigue estos pasos:
\begin{enumerate}
	\item Toca la pestaña “Área personal” en la barra de navegación inferior.
	\item Se abrirá una pantalla con tus datos personales y varios botones de acción.
	\item Toca el botón “Editar perfil” para acceder al formulario de edición de datos personales.
	\item Completa el formulario con los nuevos datos y toca el botón “Guardar cambios” para enviar el formulario.
	\item Para modificar la contraseña o eliminar la cuenta, toca la pestaña “Ajustes” dentro de la pantalla “Área personal”.
	\item Toca el botón “Cambiar contraseña” para acceder al formulario de cambio de contraseña.
	\item Completa el formulario con los nuevos datos y toca el botón “Guardar cambios” para enviar el formulario.
	\item Toca el botón “Eliminar cuenta” para eliminar tu cuenta de la aplicación.
\end{enumerate}

\subsubsection{Cerrar Sesión}
Para cerrar sesión desde tu dispositivo móvil, sigue estos pasos:
\begin{enumerate}
	\item Toca la pestaña “Área personal” en la barra de navegación inferior.
	\item Se abrirá una pantalla con tus datos personales y varios botones de acción.
	\item Toca el botón “Cerrar sesión” para salir de tu cuenta.
	\item Se te redirigirá a la pantalla de inicio y habrás cerrado sesión con éxito.
\end{enumerate}
Si deseas volver a iniciar sesión, toca la pestaña “Cuenta” en la barra de navegación inferior y sigue los pasos descritos en la sección “Inicio de Sesión” .

\subsubsection{Cambiar Idioma}
Para cambiar el idioma de la aplicación desde tu dispositivo móvil, sigue estos pasos:
\begin{enumerate}
	\item Toca la pestaña “Área personal” en la barra de navegación inferior.
	\item Toca la pestaña “Ajustes” dentro de la pantalla “Área personal”.
	\item Toca la opción “Español” .
	\item Se abrirá una lista de idiomas disponibles.
	\item Toca el idioma deseado para cambiar el idioma de la aplicación.
	\item La aplicación se actualizará automáticamente con el idioma seleccionado.
\end{enumerate}

\subsubsection{Cambiar Tema}
\begin{enumerate}
	\item Toca la pestaña “Área personal” en la barra de navegación inferior.
	\item Toca la pestaña “Ajustes” dentro de la pantalla “Área personal”.
	\item Toca la opción “Cambiar tema” .
	\item La aplicación se actualizará automáticamente con el tema contrario. Si el tema actual es claro, se cambiará a oscuro y viceversa.
	\item El icono de la opción de tema cambiará al tema contrario. Si el tema actual es claro, se cambiará a una luna y si el tema actual es oscuro, se cambiará a un sol.
\end{enumerate}

\subsubsection{Soporte Técnico}
Si encuentras algún problema o tienes alguna pregunta relacionada con el uso de nuestra aplicación, no dudes en contactar a nuestro equipo de soporte técnico. Puedes encontrar la información de contacto en el apartado “Contacto” en la sección de “Configuración” de la aplicación.