Este manual tiene como objetivo proporcionarte una guía detallada sobre cómo utilizar todas las características y funcionalidades de nuestro sitio web.

\subsubsection{Requisitos del Sistema}
Antes de comenzar, asegúrate de que tu sistema cumple con los siguientes requisitos:

\begin{itemize}
	\item Navegador web actualizado (recomendamos Google Chrome o Mozilla Firefox).
	\item Conexión a Internet estable.
	\item Resolución de pantalla mínima de 1024x768 píxeles.
\end{itemize}

\subsubsection{Acceso a la Página Web}
Para acceder a nuestra página web, sigue estos pasos:

\begin{enumerate}
	\item Abre tu navegador web.
	\item En la barra de direcciones, ingresa la URL de nuestro sitio web: www.astour.online.
	\item Presiona Enter.
\end{enumerate}

\subsubsection{Funcionalidades Principales}
A continuación, describiremos las funcionalidades principales de nuestra página web:

\subsubsection{Inicio de Sesión}
Antes de iniciar sesión deberás solicitar tus credenciales de acceso al administrador del sistema.
Con las credenciales de acceso, sigue estos pasos para iniciar sesión:

\begin{enumerate}
	\item Haz clic en la opción “Cuenta” del menú superior de la página de inicio.
	\item Ingresa tu email y contraseña en los campos correspondientes.
	\item Haz clic en el botón “Iniciar Sesión” para acceder a tu cuenta.
\end{enumerate}

Al iniciar sesión con éxito, accederás a tu cuenta y podrás comenzar a gestionar el sistema.

\subsubsection{Exploración de actividades}
Desde la página de inicio, puedes explorar las actividades pulsando en el botón “Empeza a buscar” . Te llevará a una página donde encontrarás una lista de actividades disponibles, con la opción de filtrar por nombre, por rango de fechas, precio máximo, número de personas, puntuación mínima y idioma.
\\[1ex]
Si deseas filtrar por nombre, ingresa el nombre de la actividad en la barra de búsqueda superior. A los 3 segundos de haber dejado de escribir, se mostrarán las actividades que coincidan con el nombre ingresado.
\\[1ex]
En el caso de que desee aplicar alguno de los filtros del margen izquierdo, se deberá seleccionar los filtros y, posteriormente, hacer clic en el botón “Añadir filtros” que se encuentra en la parte inferior de los filtros.
\\[1ex]
Si deseas filtrar por rango de fechas, selecciona las fechas de inicio y fin en los campos correspondientes.
Si deseas filtrar por precio máximo, selecciona el precio máximo moviendo para la izquierda o derecha el control deslizante.
Si deseas filtrar por número de personas, selecciona el número de personas haciendo uso de los botones de incremento y decremento.
Si deseas filtrar por puntuación mínima, selecciona la puntuación mínima haciendo uso de los botones de incremento y decremento.
Si deseas filtrar por idioma, selecciona los idiomas deseados haciendo clic en los cuadrados correspondientes.
\\[1ex]
En el caso de querer modificar los filtros, deberá de seleccionar los filtros a aplicar y volver a seleccionar el botón “Añadir filtros” .
\\[1ex]
En el caso de querer eliminar todos los filtros, deberá de hacer clic en el botón “Eliminar filtros” que se encuentra en la parte inferior de los filtros, justo encima del botón “Añadir filtros”.


\subsubsection{Ver información detallada de una actividad}
Para ver la información detallada de una actividad deberás hacer clic en el botón “Ver más” de la actividad deseada.
Se abrirá una nueva ventana con la información detallada de la actividad, incluyendo el nombre, la descripción, la duración, la puntuación y la ubicación.

Para poder ver los precios, idiomas y horarios de la actividad, deberás hacer clic en el botón “Ver eventos” que se encuentra bajo el nombre y la ubicación de la actividad.

Desde esta ventana podrás seleccionar la fecha en el calendario y el numero de personas haciendo uso de los botones de incremento y decremento.
Una vez seleccionada la fecha y el número de personas, podrás ver los precios, idiomas y horarios disponibles para la actividad.

\subsubsection{Panel de contol}
Una vez has iniciado sesión, podrás acceder a tu panel de control haciendo clic en la opción “Panel de control” del menú superior de la página de inicio.
En tu panel de control, podrás ver estadísticas sobre las reservas realizadas, los beneficios obtenidos, el número de usuarios registrados, reservas recientes y gráficos informativos.

En esta página también se podrá ir a la sección de usuarios y actividades para poder gestionarlos.

\subsubsection{Gestión de usuarios}
Para gestionar los usuarios, una vez has iniciado sesión, sigue estos pasos:
\begin{enumerate}
	\item Haz clic en la opción “Panel de control” del menú superior de la página de inicio.
	\item Se abrirá una página con estadísticas y un menú en la parte izquierda.
	\item Haz clic en la opción “Usuarios” del menú, para acceder a la lista de usuarios registrados.
	\item Se abrirá una página con la lista de usuarios registrados y varios botones de acción.
	\item Se pueden filtrar los usuarios por nombre, email o número de identificación, haciendo uso de la barra de búsqueda.
	\item Haz clic en el icono del ojo para acceder a la información detallada del usuario.
	\item Haz clic en el icono del lápiz para editar el usuario.
	\item Haz click en el icono de la papelera para eliminar el usuario de la aplicación.
	\item En la parte superior derecha de la lista de usuarios, se encuentra un botón para añadir un nuevo usuario.
\end{enumerate}

\subsubsection{Gestión de actividades}
Para gestionar las actividades, una vez has iniciado sesión, sigue estos pasos:
\begin{enumerate}
	\item Haz clic en la opción “Panel de control” del menú superior de la página de inicio.
	\item Se abrirá una página con estadísticas y un menú en la parte izquierda.
	\item Haz clic en la opción “Actividades” del menú, para acceder a la lista de actividades registradas.
	\item Se abrirá una página con la lista de actividades registradas y varios botones de acción.
	\item Se pueden filtrar las actividades por nombre o ubicación, haciendo uso de la barra de búsqueda.
	\item Haz clic en el icono del ojo para acceder a la información detallada de la actividad.
	\item Haz clic en el icono del lápiz para editar la actividad.
	\item Haz click en el icono de la papelera para eliminar la actividad de la aplicación y todos sus eventos asociados.
	\item En el menú de la parte izquierda, se encuentra:
	      \begin{itemize}
		      \item un botón para activar o desactivar la visibilidad de los eventos de las actividades.
		      \item un rango de fechas para filtrar los eventos por fecha de inicio.
		      \item un campo para mostrar o ocultar los eventos cancelados.
		      \item un botón para añadir un nuevo evento al sistema.
	      \end{itemize}
\end{enumerate}

\subsubsection{Cambiar datos personales}
Para cambiar tus datos personales, una vez has iniciado sesión, sigue estos pasos:
\begin{enumerate}
	\item Haz clic en la opción “Área personal” del menú superior de la página de inicio.
	\item Se abrirá una página con tus datos personales y varios botones de acción.
	\item Haz clic en el botón “Editar perfil” para acceder al formulario de edición de datos personales.
	\item Haz clic en el botón “Cambiar contraseña” para acceder al formulario de cambio de contraseña.
	\item Haz clic en el botó “Eliminar cuenta” para eliminar tu cuenta de la aplicación.
	\item Completa el formulario con los nuevos datos y haz clic en el botón “Guardar cambios” para enviar el formulario.
\end{enumerate}

\subsubsection{Cerrar Sesión}
Para cerrar sesión, sigue estos pasos:
\begin{enumerate}
	\item Haz clic en la opción “Área personal” del menú superior de la página de inicio.
	\item Se abrirá una página con tus datos personales y varios botones de acción.
	\item Haz clic en el botón “Cerrar sesión” para salir de tu cuenta.
	\item Se te redirigirá a la página de inicio y habrás cerrado sesión con éxito.
\end{enumerate}
Si deseas volver a iniciar sesión, haz clic en la opción “Cuenta” del menú superior y sigue los pasos descritos en la sección “Inicio de Sesión” .

\subsubsection{Cambiar Idioma}
Para cambiar el idioma de la página web, sigue estos pasos:
\begin{enumerate}
	\item Haz clic en la opción “Español” del menú superior de la página de inicio.
	\item Se abrirá un menú desplegable con una lista de idiomas disponibles.
	\item Haz clic en el idioma deseado para cambiar el idioma de la página web.
	\item La página web se actualizará automáticamente con el idioma seleccionado.
	\item El nombre de la opción de idioma, del menú superior, cambiará al idioma seleccionado.
\end{enumerate}

\subsubsection{Cambiar Tema}
\begin{enumerate}
	\item Haz clic en la opción “Cambiar tema” del menú superior de la página de inicio.
	\item La página web se actualizará automáticamente con el tema contrario. Si el tema actual es claro, se cambiará a oscuro y viceversa.
	\item El icono de la opción de tema, del menú superior, cambiará al tema contrario. Si el tema actual es claro, se cambiará a una luna y si el tema actual es oscuro, se cambiará a un sol.
\end{enumerate}

\subsubsection{Soporte Técnico}
Si encuentras algún problema o tienes alguna pregunta relacionada con el uso de nuestra página web, no dudes en contactar a nuestro equipo de soporte técnico. Puedes encontrar la información de contacto en el apartado “Contacto” de la parte inferior de la página.