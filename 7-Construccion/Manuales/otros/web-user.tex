Este manual tiene como objetivo proporcionarte una guía detallada sobre cómo utilizar todas las características y funcionalidades de nuestro sitio web.

\subsubsection{Requisitos del Sistema}
Antes de comenzar, asegúrate de que tu sistema cumple con los siguientes requisitos:

\begin{itemize}
	\item Navegador web actualizado (recomendamos Google Chrome o Mozilla Firefox).
	\item Conexión a Internet estable.
	\item Resolución de pantalla mínima de 1024x768 píxeles.
\end{itemize}

\subsubsection{Acceso a la Página Web}
Para acceder a nuestra página web, sigue estos pasos:

\begin{enumerate}
	\item Abre tu navegador web.
	\item En la barra de direcciones, ingresa la URL de nuestro sitio web: www.astour.online.
	\item Presiona Enter.
\end{enumerate}

\subsubsection{Funcionalidades Principales}
A continuación, describiremos las funcionalidades principales de nuestra página web:

\subsubsection{Registro de Usuario}
Para registrarte en nuestro sitio web, sigue estos pasos:

\begin{enumerate}
	\item Haz clic en la opción “Cuenta” del menú superior de la página de inicio. Se abrirá un modal con un formulario para iniciar sesión y un enlace en la parte inferior para registrarte como nuevo usuario.
	\item Haz clic en el enlace “Registrarse” para acceder al formulario de registro.
	\item Completa el formulario de registro con tu información personal.
	\item Haz clic en el botón “Registrarse” para enviar el formulario.
\end{enumerate}

Al completar el registro con éxito, accederás automáticamente a tu cuenta y podrás comenzar a reservas actividades.

\subsubsection{Inicio de Sesión}
Si ya tienes una cuenta, puedes iniciar sesión siguiendo estos pasos:

\begin{enumerate}
	\item Haz clic en la opción “Cuenta” del menú superior de la página de inicio.
	\item Ingresa tu email y contraseña en los campos correspondientes.
	\item Haz clic en el botón “Iniciar Sesión” para acceder a tu cuenta.
\end{enumerate}

Al iniciar sesión con éxito, accederás a tu cuenta y podrás comenzar a reservar actividades.

\subsubsection{Exploración de actividades}
Desde la página de inicio, puedes explorar las actividades pulsando en el botón “Empeza a buscar” . Te llevará a una página donde encontrarás una lista de actividades disponibles, con la opción de filtrar por nombre, por rango de fechas, precio máximo, número de personas, puntuación mínima y idioma.

Si deseas filtrar por nombre, ingresa el nombre de la actividad en la barra de búsqueda superior. A los 3 segundos de haber dejado de escribir, se mostrarán las actividades que coincidan con el nombre ingresado.

En el caso de que desee aplicar alguno de los filtros del margen izquierdo, se deberá seleccionar los filtros y, posteriormente, hacer clic en el botón “Añadir filtros” que se encuentra en la parte inferior de los filtros.

Si deseas filtrar por rango de fechas, selecciona las fechas de inicio y fin en los campos correspondientes.
Si deseas filtrar por precio máximo, selecciona el precio máximo moviendo para la izquierda o derecha el control deslizante.
Si deseas filtrar por número de personas, selecciona el número de personas haciendo uso de los botones de incremento y decremento.
Si deseas filtrar por puntuación mínima, selecciona la puntuación mínima haciendo uso de los botones de incremento y decremento.
Si deseas filtrar por idioma, selecciona los idiomas deseados haciendo clic en los cuadrados correspondientes.

En el caso de querer modificar los filtros, deberá de seleccionar los filtros a aplicar y volver a seleccionar el botón “Añadir filtros” .

En el caso de querer eliminar todos los filtros, deberá de hacer clic en el botón “Eliminar filtros” que se encuentra en la parte inferior de los filtros, justo encima del botón “Añadir filtros”.


\subsubsection{Ver información detallada de una actividad}
Para ver la información detallada de una actividad deberás hacer clic en el botón “Ver más” de la actividad deseada.
Se abrirá una nueva ventana con la información detallada de la actividad, incluyendo el nombre, la descripción, la duración, la puntuación y la ubicación.

Para poder ver los precios, idiomas y horarios de la actividad, deberás hacer clic en el botón “Ver disponibilidad” que se encuentra bajo el nombre y la ubicación de la actividad.

Desde esta ventana podrás seleccionar la fecha en el calendario y el numero de personas haciendo uso de los botones de incremento y decremento.
Una vez seleccionada la fecha y el número de personas, podrás ver los precios, idiomas y horarios disponibles para la actividad.

\subsubsection{Reservar una actividad}
Para reservar una actividad, sigue estos pasos:

\begin{enumerate}
	\item Explora las actividades disponibles y selecciona la que deseas reservar.
	\item Haz clic en el botón “Ver más” de la actividad deseada.
	\item Se abrirá una nueva ventana con la información detallada de la actividad.
	\item Haz clic en el botón “Ver disponibilidad” para seleccionar la fecha y el número de personas.
	\item Selecciona la fecha en el calendario y el número de personas haciendo uso de los botones de incremento y decremento.
	\item Selecciona el horario y el idioma deseados dentro de las opciones disponibles.
	\item Haz clic en el botón “Reservar” para confirmar la reserva. Si no has iniciado sesión, se te pedirá que inicies sesión o te registres.
	\item Se te llevará a una página de confirmación de la reserva, donde verás tus datos personales (se pueden modificar si es necesario), el precio total y un formulario para introducir los datos de pago.
	\item Completa el formulario con los datos de pago y haz clic en el botón “Pagar” para finalizar la reserva.
\end{enumerate}

\subsubsection{Gestión de Reservas}
Para gestionar tus reservas, sigue estos pasos:
\begin{enumerate}
	\item Haz clic en la opción “Reservas” del menú superior de la página de inicio.
	\item Se abrirá una página con un listado de tus reservas activas y pasadas.
	\item Para ver más detalles de una reserva, haz clic en el botón “Gestionar” de la reserva deseada.
	\item Se abrirá una nueva ventana con la información detallada de la reserva, incluyendo la actividad, la fecha y hora, el idioma, el número de personas y el precio total.
	\item Para cancelar una reserva, haz clic en el botón “Cancelar” de la reserva deseada.
	\item Se abrirá una ventana de confirmación de la cancelación. Haz clic en el botón “Confirmar” para cancelar la reserva.
\end{enumerate}

\subsubsection{Publicar una valoración}
Recuerda que puedes publicar una valoración de una actividad después de haberla realizado.

Para publicar una valoración de una actividad, sigue estos pasos:
\begin{enumerate}
	\item Haz clic en la opción “Reservas” del menú superior de la página de inicio.
	\item Se abrirá una página con un listado de tus reservas activas y pasadas.
	\item Haz clic en el botón “Gestionar” de la reserva a la que deseas valorar.
	\item Se abrirá una nueva ventana con la información detallada de la reserva, incluyendo la actividad, la fecha y hora, el idioma, el número de personas y el precio total.
	\item Para publicar una valoración, haz clic en el botón “Añadir valoración” de la reserva deseada.
	\item Se abrirá una nueva ventana con un formulario para introducir tu puntuación y comentario (campo no obligatorio) de la actividad.
	\item Completa el formulario y haz clic en el botón “Guardar cambios” para enviar la valoración.
\end{enumerate}

\subsubsection{Cambiar datos personales}
Para cambiar tus datos personales, una vez has iniciado sesión, sigue estos pasos:
\begin{enumerate}
	\item Haz clic en la opción “Área personal” del menú superior de la página de inicio.
	\item Se abrirá una página con tus datos personales y varios botones de acción.
	\item Haz clic en el botón “Editar perfil” para acceder al formulario de edición de datos personales.
	\item Haz clic en el botón “Cambiar contraseña” para acceder al formulario de cambio de contraseña.
	\item Haz clic en el botó “Eliminar cuenta” para eliminar tu cuenta de la aplicación.
	\item Completa el formulario con los nuevos datos y haz clic en el botón “Guardar cambios” para enviar el formulario.
\end{enumerate}

\subsubsection{Cerrar Sesión}
Para cerrar sesión, sigue estos pasos:
\begin{enumerate}
	\item Haz clic en la opción “Área personal” del menú superior de la página de inicio.
	\item Se abrirá una página con tus datos personales y varios botones de acción.
	\item Haz clic en el botón “Cerrar sesión” para salir de tu cuenta.
	\item Se te redirigirá a la página de inicio y habrás cerrado sesión con éxito.
\end{enumerate}
Si deseas volver a iniciar sesión, haz clic en la opción “Cuenta” del menú superior y sigue los pasos descritos en la sección “Inicio de Sesión” .

\subsubsection{Cambiar Idioma}
Para cambiar el idioma de la página web, sigue estos pasos:
\begin{enumerate}
	\item Haz clic en la opción “Español” del menú superior de la página de inicio.
	\item Se abrirá un menú desplegable con una lista de idiomas disponibles.
	\item Haz clic en el idioma deseado para cambiar el idioma de la página web.
	\item La página web se actualizará automáticamente con el idioma seleccionado.
	\item El nombre de la opción de idioma, del menú superior, cambiará al idioma seleccionado.
\end{enumerate}

\subsubsection{Cambiar Tema}
\begin{enumerate}
	\item Haz clic en la opción “Cambiar tema” del menú superior de la página de inicio.
	\item La página web se actualizará automáticamente con el tema contrario. Si el tema actual es claro, se cambiará a oscuro y viceversa.
	\item El icono de la opción de tema, del menú superior, cambiará al tema contrario. Si el tema actual es claro, se cambiará a una luna y si el tema actual es oscuro, se cambiará a un sol.
\end{enumerate}

\subsubsection{Soporte Técnico}
Si encuentras algún problema o tienes alguna pregunta relacionada con el uso de nuestra página web, no dudes en contactar a nuestro equipo de soporte técnico. Puedes encontrar la información de contacto en el apartado “Contacto” de la parte inferior de la página.