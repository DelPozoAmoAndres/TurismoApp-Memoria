\chapter{Construcción del sistema de información}
\section{Preparación del entorno de generación y construcción}
Esta sección aborda diversos aspectos relacionados con la implementación del software desarrollado para la aplicación de turismo.
\subsection{Estándares y normas seguidas}
Durante el desarrollo de esta aplicación, se han seguido una serie de estándares y normativas para garantizar tanto la calidad del software como el cumplimiento legal y ético. Entre los estándares más relevantes se encuentran:
\begin {itemize}
\item {\bfseries ECMAScript:} Este estándar proporciona una base sólida para el desarrollo en Javascript y Typescript, asegurando compatibilidad y comprensión del código entre diferentes desarrolladores.
\item {\bfseries Protocolo HTTPS:} Se ha utilizado este protocolo para garantizar la seguridad y eficiencia en la comunicación entre la aplicación y el servidor.
\item {\bfseries Principios REST:} Se ha seguido el estilo arquitectónico REST para el diseño de la API, garantizando una comunicación eficiente y escalable.
\item {\bfseries Reglamento General de Protección de Datos (RGPD):} Se han implementado medidas para asegurar que la aplicación cumple con las normativas europeas de protección de datos, garantizando la privacidad y seguridad de los datos personales de los usuarios.
\item {\bfseries Ley de Servicios de la Sociedad de la Información (LSSI):} Se ha tenido en cuenta esta ley para asegurar el cumplimiento de las normativas españolas sobre comercio electrónico y servicios en línea.
\item {\bfseries Material Design:} Se ha seguido este estándar de diseño para garantizar una interfaz de usuario intuitiva y atractiva.
\end {itemize}

Además, para la implementación del proyecto, se ha recurrido a varios recursos de apoyo, como tutoriales oficiales, documentación técnica y materiales de asignaturas específicas del grado. Entre las fuentes más consultadas se encuentran:
\begin {itemize}
\item {\bfseries Documentación oficial de React y Ionic:} Guías y tutoriales oficiales que han proporcionado una comprensión profunda de estas tecnologías.
\item {\bfseries Cursos en línea y MOOC:}  Recursos adicionales que han facilitado la adopción de prácticas avanzadas en Typescript y Express.
\end {itemize}

\subsection{Lenguajes de programación}
Para la implementación de la aplicación se han utilizado varios lenguajes de programación, cada uno con un propósito específico. Los lenguajes más relevantes son:
\begin {itemize}
\item {\bfseries Typescript:} Se ha utilizado Typescript para el desarrollo tanto del frontend como del backend de la aplicación. Typescript es un lenguaje de programación que añade tipado estático a Javascript, lo que facilita la detección de errores y la escritura de código más robusto y mantenible.
\item {\bfseries HTML y CSS:} Se han utilizado HTML y CSS para la maquetación y estilización de la interfaz de usuario de la aplicación.
\item {\bfseries Latex:} Se ha utilizado Latex para la redacción de la documentación del proyecto, incluyendo la memoria y los anexos.
\item {\bfseries PlantUML:} Se ha utilizado PlantUML para la generación de diagramas UML que han facilitado la comprensión y documentación del diseño de la aplicación.
\end {itemize}

\subsection{Frameworks y librerías}
Para el desarrollo de la aplicación se han utilizado varios frameworks y librerías que han facilitado la implementación de funcionalidades específicas. Algunos de los frameworks y librerías más relevantes son:
\begin {itemize}
\item {\bfseries React:} Se ha utilizado React para el desarrollo del frontend de la aplicación. React es una biblioteca de Javascript que permite la creación de interfaces de usuario interactivas y reactivas.
\item {\bfseries Ionic:} Se ha utilizado Ionic para el desarrollo de la aplicación móvil. Ionic es un framework de código abierto que permite la creación de aplicaciones móviles multiplataforma utilizando tecnologías web como HTML, CSS y Typescript.
\item {\bfseries Express:} Se ha utilizado Express para el desarrollo del backend de la aplicación. Express es un framework de Node.js que facilita la creación de servidores web y APIs REST.
\item {\bfseries Mongoose:} Se ha utilizado Mongoose para la definición y gestión de los modelos de datos de la aplicación. Mongoose es una biblioteca de modelado de objetos para MongoDB que facilita la interacción con la base de datos.
\item {\bfseries Axios:} Se ha utilizado Axios para la realización de peticiones HTTP desde el frontend de la aplicación. Axios es una biblioteca de Javascript que simplifica la comunicación con servidores web.
\item {\bfseries Bycript:} Se ha utilizado Bycript para el cifrado de contraseñas de usuario en la aplicación. Bycript es una biblioteca de Node.js que facilita el almacenamiento seguro de contraseñas.
\item {\bfseries JWT:} Se ha utilizado JWT para la autenticación y autorización de usuarios en la aplicación. JWT es un estándar abierto que define un método compacto y seguro para la transferencia de información entre dos partes.
\end {itemize}

\subsection{Herramientas y programas usados para el desarrollo}
Durante el desarrollo de la aplicación se han utilizado varias herramientas y programas que han facilitado la implementación y gestión del proyecto.
\begin {itemize}
\item {\bfseries Visual Studio Code:} Se ha utilizado Visual Studio Code como entorno de desarrollo integrado (IDE) para la escritura de código y documentación. Visual Studio Code es un editor de código ligero y altamente personalizable que ofrece soporte para múltiples lenguajes de programación.
\item {\bfseries Postman:} Se ha utilizado Postman para probar y depurar la API REST de la aplicación. Postman es una herramienta que permite realizar peticiones HTTP y visualizar las respuestas de forma sencilla.
\item {\bfseries Git y GitHub:} Se ha utilizado Git como sistema de control de versiones para el seguimiento de cambios en el código fuente. GitHub se ha utilizado como plataforma de alojamiento remoto para el repositorio de código.
\item {\bfseries Docker:} Se ha utilizado Docker para la creación y gestión de contenedores que han facilitado el despliegue de la aplicación en entornos de producción.
\item {\bfseries XCode y Android Studio:} Se han utilizado XCode y Android Studio para la compilación y empaquetado de la aplicación móvil para las plataformas iOS y Android, respectivamente.
\item {\bfseries MongoDB Atlas:} Se ha utilizado MongoDB Atlas como servicio de base de datos en la nube para almacenar los datos de la aplicación.
\item {\bfseries AWS E2C:} Se ha utilizado AWS E2C como servicio de alojamiento en la nube para el despliegue del servidor en entornos de producción.
\item {\bfseries Vercel:} Se ha utilizado Vercel como servicio de alojamiento en la nube para el despliegue del frontend en entornos de producción.
\end {itemize}

\section{Generación del código de los componentes}
En esta sección se describen algunos de los componentes más relevantes de la aplicación y se muestran ejemplos de código que ilustran su implementación.

Para más información sobre el código fuente de la aplicación, se puede consultar el repositorio de GitHub o el Swagger de la API REST.

\section{Ejecución de pruebas unitarias}
Se han realizado pruebas unitarias para garantizar la calidad y fiabilidad del software desarrollado. Estas pruebas han permitido detectar y corregir errores en el código y asegurar que las funcionalidades de la aplicación se comportan como se espera.
En total se han realizado 319 pruebas unitarias que cubren la mayoría de las funcionalidades de la aplicación, incluyendo la autenticación de usuarios, la gestión de actividades y usuarios y la planificación de eventos.
\begin{figure}[H]
	\centering
	\includegraphics[width=0.8\textwidth]{7-Construccion/Pruebas/unitarias.png}
	\caption{Resultado de las pruebas unitarias}
\end{figure}

\section{Ejecución de pruebas de integración}
Se han realizado pruebas de integración para comprobar que los diferentes componentes de la aplicación funcionan correctamente cuando se combinan.
En total se han realizado 22 pruebas de integración que cubren los principales escenarios de uso de la aplicación.
\begin{figure}[H]
	\centering
	\includegraphics[width=0.8\textwidth]{7-Construccion/Pruebas/integracion.png}
	\caption{Resultado de las pruebas integración}
\end{figure}
\section{Ejecución de pruebas del sistema}
\subsection{Pruebas de usabilidad}
En esta sección se presentarán los resultados obtenidos de las pruebas de usabilidad realizadas con los participantes. El objetivo es analizar la eficacia, eficiencia y satisfacción del usuario al interactuar con la aplicación.
\subsubsection{Preguntas generales}
Estos son los resultados obtenidos de las preguntas generales realizadas a los participantes:
% \begin{figure}[H]
% 	\centering
% 	\includegraphics[width=0.8\textwidth]{7-Construccion/Pruebas/Usabilidad/preguntas-generales.png}
% 	\caption{Resultados de las preguntas generales}
% \end{figure}

\subsubsection{Actividades guiadas}
Estos son los resultados obtenidos de las actividades guiadas realizadas por los participantes:
% \begin{figure}[H]
% 	\centering
% 	\includegraphics[width=0.8\textwidth]{7-Construccion/Pruebas/Usabilidad/actividades-guiadas.png}
% 	\caption{Resultados de las actividades guiadas}
% \end{figure}

\subsubsection{Preguntas cortas sobre la aplicación}
Estos son los resultados obtenidos de las preguntas cortas sobre la aplicación realizadas a los participantes:
% \begin{figure}[H]
% 	\centering
% 	\includegraphics[width=0.8\textwidth]{7-Construccion/Pruebas/Usabilidad/preguntas-cortas.png}
% 	\caption{Resultados de las preguntas cortas}
% \end{figure}

\section{Elaboración de los manuales de usuario}
\subsection{Manual de usuario}
Este manual tiene como objetivo proporcionarte una guía detallada sobre cómo utilizar todas las características y funcionalidades de nuestra aplicación.
Nuestra aplicación está diseñada para ser accesible a través de la web y desde la app móvil, por lo que habrá algunas diferencias en la interfaz y en la forma de interactuar con la aplicación dependiendo del dispositivo que utilices.

\subsubsection{Requisitos del Sistema}
Antes de comenzar, asegúrate de que tu dispositivo cumple con los siguientes requisitos:

\begin{itemize}
	\item Navegador web móvil actualizado (recomendamos Google Chrome o Mozilla Firefox) o la app instalada.
	\item Conexión a Internet estable.
	\item Resolución de pantalla mínima de 375x667 píxeles.
\end{itemize}

\subsubsection{Acceso a la Página Web}
En caso de no tener app instalada, acceder a nuestra página web siguiendo estos pasos:

\begin{enumerate}
	\item Abre tu navegador web.
	\item En la barra de direcciones, ingresa la URL de nuestro sitio web: www.astour.online.
	\item Presiona Enter.
\end{enumerate}

\subsubsection{Funcionalidades Principales}
A continuación, describiremos las funcionalidades principales:

\subsubsection{Autenticación}
\hrulefill
\subsubsection{Registro de Usuario}
Si eres un guía o un nuevo administrador pide a un administrador que te cree una cuenta.
De lo contrario, registrarte siguiendo estos pasos:

\begin{enumerate}
	\item Accede al apartado de cuenta. Mirar figuras \ref{fig:cuenta-movil}, \ref{fig:cuenta-app}, \ref{fig:cuenta-web}.
	\item Se abrirá un modal con un formulario para iniciar sesión y un enlace en la parte inferior para registrarte como nuevo usuario.
	      Haz clic sobre el enlace “Registrate aquí” para acceder al formulario de registro. Mirar figura \ref{fig:inicio-form}.
	\item Completa el formulario de registro con tu información personal y pulsa el botón “Registrarse” para enviar el formulario. Mirar figura \ref{fig:registro-form}.
\end{enumerate}

Al completar el registro con éxito, accederás automáticamente a tu cuenta.


\subsubsection{Inicio de Sesión}
Si ya tienes una cuenta, puedes iniciar sesión siguiendo estos pasos:

\begin{enumerate}
	\item Accede al apartado de cuenta. Mirar figuras \ref{fig:cuenta-movil}, \ref{fig:cuenta-app}, \ref{fig:cuenta-web}.
	\item Se abrirá un modal con un formulario para iniciar sesión.
	      Ingresa tu email y contraseña en los campos correspondientes y pulsa el botón “Iniciar Sesión” para acceder a tu cuenta. Mirar figura \ref{fig:inicio-form}.
\end{enumerate}

\begin{figure}[H]
	\centering
	\begin{minipage}{0.45\textwidth}
		\centering
		\includegraphics[width=0.45\textwidth]{7-Construccion/Manuales/mobile/menu marcado.png}
		\includegraphics[width=0.45\textwidth]{7-Construccion/Manuales/mobile/cuenta marcado.png}
		\caption{Autenticación \\ Despliegue del menú y selección de la opción “Cuenta” .}
		\label{fig:cuenta-movil}
	\end{minipage}
	\hfill
	\begin{minipage}{0.45\textwidth}
		\centering
		\includegraphics[width=0.45\textwidth]{7-Construccion/Manuales/app/P1-Registro.png}
		\caption{Autenticación \\ Selección de la opción “Cuenta” de la barra inferior de navegación.}
		\label{fig:cuenta-app}
	\end{minipage}
\end{figure}
\begin{figure}[H]
	\centering
	\includegraphics[width=0.5\textwidth]{7-Construccion/Manuales/web/cuenta opcion.png}
	\caption{Autenticación \\ Selección de la opción “Registrarse” .}
	\label{fig:cuenta-web}
\end{figure}

\begin{figure}[H]
	\centering
	\begin{minipage}{0.45\textwidth}
		\centering
		\includegraphics[width=1\textwidth]{7-Construccion/Manuales/web/modal inicio.png}
		\caption{Autenticación \\ Modal de inicio de sesión.}
		\label{fig:inicio-form}
	\end{minipage}
	\hfill
	\begin{minipage}{0.45\textwidth}
		\centering
		\includegraphics[width=1\textwidth]{7-Construccion/Manuales/web/modal registro.png}
		\caption{Autenticación \\ Modal de registro de sesión.}
		\label{fig:registro-form}
	\end{minipage}
\end{figure}


\subsubsection{Exploración de Actividades}
\hrulefill
\subsubsection{Buscar Actividades}

Desde la página de inicio, puedes buscar las actividades pulsando el botón “Empezar a buscar” . Mirar figura \ref{fig:inicio-buscar}.
\\ \\[1ex]
Te llevará a una página donde encontrarás una lista de actividades disponibles, con la opción de filtrar por nombre, por rango de fechas, precio máximo, número de personas, puntuación mínima y idioma.
Mirar figura \ref{fig:buscar-actividades}.
\\ \\[1ex]
Para filtrar por nombre, ingresa el nombre de la actividad en la barra de búsqueda superior. A los 3 segundos de haber dejado de escribir, se mostrarán las actividades que coincidan con el nombre ingresado. Mirar \ref{fig:buscar-nombre}.
\\ \\[1ex]
Para filtrar por una características más específicas, haz uso del meú de filtros. Mirar figura \ref{fig:filtros-menu}.
\\ \\[1ex]
En caso de estar desde dispositivo móvil, para llegar a ese menú se deberá pulsar el botón “Añadir filtros” que se encuentra en la parte inferior de la pantalla. Mirar figura \ref{fig:filtros-movil}.
\\ \\[1ex]
Para filtrar por rango de fechas, selecciona las fechas de inicio y fin en los campos correspondientes.
Para filtrar por precio máximo, ajusta el control deslizante.
Para filtrar por número de personas, utiliza los botones de incremento y decremento.
Para filtrar por puntuación mínima, utiliza los botones de incremento y decremento.
Para filtrar por idioma, selecciona los idiomas deseados tocando las casillas correspondientes.
\\ \\[1ex]
Para aplicar los filtros específicos, pulsa el botón “Añadir filtros” que se encuentra en la parte inferior del menú de filtros. Mirar figura \ref{fig:añadir-filtros}.
En el caso que desees modificar los filtros aplicados, vuelve a pulsar el botón “Añadir filtros” después de haber seleccionado los filtros deseados.
\\ \\[1ex]
Una vez aplicados los filtros deseados, se mostrarán las actividades que coincidan con los filtros aplicados.
\\ \\[1ex]
En caso de que desees hacer modificaciones o borrar los filtros, deberás pulsar el botón “Eliminar filtros” . Mirar figura \ref{fig:eliminar-filtros}.

\begin{figure}[H]
	\centering
	\begin{minipage}{0.45\textwidth}
		\centering
		\includegraphics[width=1\textwidth]{7-Construccion/Manuales/web/inicio buscar.png}
		\caption{Exploración de Actividades \\ Botón para buscar actividades.}
		\label{fig:inicio-buscar}
	\end{minipage}
	\hfill
	\begin{minipage}{0.45\textwidth}
		\centering
		\includegraphics[width=1\textwidth]{7-Construccion/Manuales/web/buscar actividades.png}
		\caption{Exploración de Actividades \\ Listado de actividades}
		\label{fig:buscar-actividades}
	\end{minipage}
\end{figure}

\begin{figure}[H]
	\centering
	\begin{minipage}{0.45\textwidth}
		\centering
		\includegraphics[width=1\textwidth]{7-Construccion/Manuales/web/buscar nombre.png}
		\caption{Exploración de Actividades \\ Buscar por nombre}
		\label{fig:buscar-nombre}
	\end{minipage}
	\hfill
	\begin{minipage}{0.45\textwidth}
		\centering
		\includegraphics[width=1\textwidth]{7-Construccion/Manuales/web/menu filtros.png}
		\caption{Exploración de Actividades \\ Menú de filtros}
		\label{fig:filtros-menu}
	\end{minipage}
\end{figure}


\begin{figure}[H]
	\centering
	\begin{minipage}{0.45\textwidth}
		\centering
		\includegraphics[width=0.45\textwidth]{7-Construccion/Manuales/mobile/añadir filtros.png}
		\caption{Exploración de Actividades \\ Abrir menú de filtros}
		\label{fig:filtros-movil}
	\end{minipage}
	\hfill
	\begin{minipage}{0.45\textwidth}
		\centering
		\includegraphics[width=1\textwidth]{7-Construccion/Manuales/web/añadir filtros.png}
		\caption{Exploración de Actividades \\ Añadir filtros}
		\label{fig:añadir-filtros}
	\end{minipage}
\end{figure}

\begin{figure}[H]
	\centering
	\includegraphics[width=0.5\textwidth]{7-Construccion/Manuales/web/eliminar filtros.png}
	\caption{Exploración de Actividades \\ Eliminar filtros}
	\label{fig:eliminar-filtros}
\end{figure}


\subsubsection{Ver Información Detallada de una Actividad}
Para ver la información detallada de una actividad, pulsa el botón “Ver más” de la actividad deseada.
Mirar figura \ref{fig:ver-info}.
\\ \\[1ex]
Se abrirá una nueva ventana con la información detallada de la actividad, incluyendo el nombre, la descripción, la duración, la puntuación y la ubicación.
Mirar figura \ref{fig:detalles-actividad}.
\\ \\[1ex]
Para ver los precios, idiomas y horarios de la actividad, pulsa el botón “Ver disponibilidad” que se encuentra bajo la descripción de la actividad.
Mirar figura \ref{fig:ver-disponibilidad}.
\\ \\[1ex]
Podrás seleccionar la fecha en el calendario y el número de personas usando los botones de incremento y decremento.
Una vez seleccionada la fecha y el número de personas, podrás ver los precios, idiomas y horarios disponibles para la actividad.
Mirar figura \ref{fig:disponibilidad}.


\begin{figure}[H]
	\centering
	\includegraphics[width=0.5\textwidth]{7-Construccion/Manuales/web/ver info.png}
	\caption{Exploración de Actividades \\ Ver información detallada}
	\label{fig:ver-info}
\end{figure}

\begin{figure}[H]
	\centering
	\begin{minipage}{0.45\textwidth}
		\centering
		\includegraphics[width=1\textwidth]{7-Construccion/Manuales/web/detalles actividad.png}
		\caption{Exploración de Actividades \\ Detalles de la actividad}
		\label{fig:detalles-actividad}
	\end{minipage}
	\hfill
	\begin{minipage}{0.45\textwidth}
		\centering
		\includegraphics[width=1\textwidth]{7-Construccion/Manuales/web/ver disponibilidad.png}
		\caption{Exploración de Actividades \\ Ver disponibilidad}
		\label{fig:ver-disponibilidad}
	\end{minipage}
\end{figure}

\begin{figure}[H]
	\centering
	\includegraphics[width=0.4\textwidth]{7-Construccion/Manuales/web/disponibilidad sin.png}
	\includegraphics[width=0.4\textwidth]{7-Construccion/Manuales/web/disponibilidad selec dia.png}
	\caption{Exploración de Actividades \\ Calendario de disponibilidad}
	\label{fig:disponibilidad}
\end{figure}

\subsubsection{Cambiar Datos Personales}
Para cambiar tus datos personales, una vez hayas iniciado sesión, sigue estos pasos:
\begin{enumerate}
	\item Accede al apartado “Área personal” . Mira figuras \ref{fig:areaPersonal-movil}, \ref{fig:areaPersonal-app}, \ref{fig:areaPersonal-web}.

	\item Se abrirá una página con tus datos personales y varios botones de acción. Pulsa el botón “Editar perfil” para acceder al formulario de edición de datos personales.
	      Mirar figura \ref{fig:areaPersonal-editar}.
	\item Completa el formulario con los nuevos datos y pulsar el botón “Guardar cambios” para enviar el formulario.
	\item Si se quiere cambiar la contraseña habrá que pulsar el botón “Cambiar contraseña” para acceder al formulario de cambio de contraseña, completa el formulario con los nuevos datos y pulsa el botón “Guardar cambios” para enviar el formulario.
	      En versión móvil será necesario pulsar la pestaña “Cuenta” para acceder a la opción de cambiar la contraseña. Mirar figura \ref{fig:areaPersonal-cuenta-movil} y \ref{fig:areaPersonal-cambiar-contraseña}.
	\item En el caso de querer eliminar la cuenta habrá que pulsar el botón “Eliminar cuenta” para acceder a la ventana de confirmación de la eliminación de la cuenta. Toca el botón “Confirmar” para eliminar tu cuenta.
	      En versión móvil será necesario pulsar la pestaña “Cuenta” para acceder a la opción de eliminar la cuenta. Mirar figura \ref{fig:areaPersonal-cuenta-movil} y \ref{fig:areaPersonal-eliminar}.
\end{enumerate}

\begin{figure}[H]
	\centering
	\begin{minipage}{0.45\textwidth}
		\centering
		\includegraphics[width=0.3\textwidth]{7-Construccion/Manuales/mobile/menu marcado.png}
		\includegraphics[width=0.3\textwidth]{7-Construccion/Manuales/mobile/area personal marcado.png}
		\caption{Perfil \\ Despliegue del menú y selección de la opción “Área personal” .}
		\label{fig:areaPersonal-movil}
	\end{minipage}
	\hfill
	\begin{minipage}{0.45\textwidth}
		\centering
		\includegraphics[width=0.3\textwidth]{7-Construccion/Manuales/app/P1-Perfil.png}
		\caption{Perfil \\ Selección de la opción “Perfil” de la barra inferior de navegación.}
		\label{fig:areaPersonal-app}
	\end{minipage}
\end{figure}

\begin{figure}[H]
	\centering
	\includegraphics[width=0.5\textwidth]{7-Construccion/Manuales/web/area personal opcion.png}
	\caption{Perfil \\ Selección de la opción “Área personal” de la barra superior de navegación.}
	\label{fig:areaPersonal-web}
\end{figure}

\begin{figure}[H]
	\centering
	\begin{minipage}{0.45\textwidth}
		\centering
		\includegraphics[width=0.3\textwidth]{7-Construccion/Manuales/mobile/editar datos.png}
		\includegraphics[width=0.3\textwidth]{7-Construccion/Manuales/mobile/formulario perfil.png}
		\caption{Perfil \\ Edición de datos personales.}
		\label{fig:areaPersonal-editar}
	\end{minipage}
	\hfill
	\begin{minipage}{0.45\textwidth}
		\centering
		\includegraphics[width=0.3\textwidth]{7-Construccion/Manuales/mobile/apartado cuenta seleccionado.png}
		\includegraphics[width=0.3\textwidth]{7-Construccion/Manuales/mobile/editar cuenta.png}
		\caption{Perfil \\ Apartado cuenta}
		\label{fig:areaPersonal-cuenta-movil}
	\end{minipage}
\end{figure}

\begin{figure}[H]
	\centering
	\begin{minipage}{0.45\textwidth}
		\centering
		\includegraphics[width=0.3\textwidth]{7-Construccion/Manuales/mobile/cambiar contraseña.png}
		\includegraphics[width=0.3\textwidth]{7-Construccion/Manuales/mobile/formulario contraseña.png}
		\caption{Perfil \\ Cambio de contraseña}
		\label{fig:areaPersonal-cambiar-contraseña}
	\end{minipage}
	\hfill
	\begin{minipage}{0.45\textwidth}
		\centering
		\includegraphics[width=0.3\textwidth]{7-Construccion/Manuales/mobile/eliminar cuenta.png}
		\includegraphics[width=0.3\textwidth]{7-Construccion/Manuales/mobile/confirmar eliminacion cuenta.png}
		\caption{Perfil \\ Eliminar cuenta}
		\label{fig:areaPersonal-eliminar}
	\end{minipage}
\end{figure}


\subsubsection{Cerrar Sesión}
Para poder cerrar sesión debes estar previamente logueado en la aplicación. Si no lo estás deberás hacerlo siguiendo los pasos descritos en la sección “Inicio de Sesión” .
\\ \\[1ex]
Para cerrar sesión debes buscar la opción “Cerrar sesión” . Mirar figuras \ref{fig:cuenta-movil}, \ref{fig:cuenta-app}, \ref{fig:cuenta-web}.
\\ \\[1ex]
Se te redirigirá a la página de inicio y habrás cerrado sesión con éxito. Si deseas volver a iniciar sesión, sigue los pasos descritos en la sección “Inicio de Sesión” .


\begin{figure}[H]
	\centering
	\begin{minipage}{0.45\textwidth}
		\centering
		\includegraphics[width=0.3\textwidth]{7-Construccion/Manuales/mobile/menu marcado.png}
		\includegraphics[width=0.3\textwidth]{7-Construccion/Manuales/mobile/cerrar sesion marcado.png}
		\caption{Cerrar sesión - Despliegue del menú y selección de la opción “Cerrar sesión” .}
		\label{fig:cuenta-movil}
	\end{minipage}
	\hfill
	\begin{minipage}{0.45\textwidth}
		\centering
		\includegraphics[width=0.3\textwidth]{7-Construccion/Manuales/app/P1-Perfil.png}
		\includegraphics[width=0.3\textwidth]{7-Construccion/Manuales/app/P2-CerrarSesion.png}
		\caption{Cerrar sesión - Cerrar sesión desde la app.}
		\label{fig:cuenta-app}
	\end{minipage}
\end{figure}

\begin{figure}[H]
	\centering
	\includegraphics[width=0.5\textwidth]{7-Construccion/Manuales/web/cerrar sesion.png}
	\caption{Cerrar sesión - Cerrar sesión desde la web.}
	\label{fig:cuenta-web}
\end{figure}

\subsubsection{Cambiar Idioma}
Para cambiar el idioma de la página sigue estos pasos:
\begin{enumerate}
	\item Busca el botón “Español” . El contenido del botón puede variar dependiendo del idioma actual. Mirar figura \ref{fig:idioma-movil}, \ref{fig:idioma-app}, \ref{fig:idioma-web}.

	\item Se abrirá un menú desplegable con una lista de idiomas disponibles. Toca el idioma deseado para cambiar el idioma de la aplicación.
	      El contenido de la aplicación se actualizará automáticamente con el idioma seleccionado. Mirar figura \ref{fig:cambio-idioma}.

\end{enumerate}

\begin{figure}[H]
	\centering
	\begin{minipage}{0.45\textwidth}
		\centering
		\includegraphics[width=0.3\textwidth]{7-Construccion/Manuales/mobile/menu marcado.png}
		\includegraphics[width=0.3\textwidth]{7-Construccion/Manuales/mobile/idioma marcado.png}
		\caption{Cambiar idioma - Despliegue del menú y selección de la opción “Español” .}
		\label{fig:idioma-movil}
	\end{minipage}
	\hfill
	\begin{minipage}{0.45\textwidth}
		\centering
		\includegraphics[width=0.3\textwidth]{7-Construccion/Manuales/app/P1-Configuration.png}
		\caption {Cambiar idioma - Ir a los ajustes.}
		\label{fig:idioma-app}
	\end{minipage}
\end{figure}

\begin{figure}[H]
	\centering
	\begin{minipage}{0.45\textwidth}
		\centering
		\includegraphics[width=1\textwidth]{7-Construccion/Manuales/web/idioma.png}
		\caption{Cambiar idioma - Cambio de idioma.}
		\label{fig:idioma-web}
	\end{minipage}
	\begin{minipage}{0.45\textwidth}
		\centering
		\includegraphics[width=0.3\textwidth]{7-Construccion/Manuales/mobile/opciones idioma.png}
		\includegraphics[width=0.3\textwidth]{7-Construccion/Manuales/mobile/ingles.png}
		\caption{Cambiar idioma - Cambio de idioma.}
		\label{fig:cambio-idioma}
	\end{minipage}
\end{figure}

\subsubsection{Cambiar Tema}
\begin{enumerate}
	\item Busca el botón “Cambiar tema” . Mirar figura \ref{fig:tema-movil}, \ref{fig:tema-app}, \ref{fig:tema-web}.

	\item La página web se actualizará automáticamente con el tema contrario. Si el tema actual es claro, se cambiará a oscuro y viceversa.
	      El icono de la opción de tema cambiará al tema contrario. Si el tema actual es claro, se cambiará a una luna y si el tema actual es oscuro, se cambiará a un sol.
	      Mirar figura \ref{fig:cambio-tema}.

\end{enumerate}

\begin{figure}[H]
	\centering
	\begin{minipage}{0.45\textwidth}
		\centering
		\includegraphics[width=0.3\textwidth]{7-Construccion/Manuales/mobile/menu marcado.png}
		\includegraphics[width=0.3\textwidth]{7-Construccion/Manuales/mobile/tema marcado.png}
		\caption{Cambiar tema - Despliegue del menú y selección de la opción “Cambiar tema” .}
		\label{fig:tema-movil}
	\end{minipage}
	\hfill
	\begin{minipage}{0.45\textwidth}
		\centering
		\includegraphics[width=0.3\textwidth]{7-Construccion/Manuales/app/P1-Configuration.png}
		\caption{Cambiar tema - Ir a los ajustes.}
		\label{fig:tema-app}
	\end{minipage}
\end{figure}

\begin{figure}[H]
	\centering
	\begin{minipage}{0.45\textwidth}
		\centering
		\includegraphics[width=1\textwidth]{7-Construccion/Manuales/web/tema.png}
		\caption{Cambiar tema - Cambio de tema.}
		\label{fig:tema-web}
	\end{minipage}
	\hfill
	\begin{minipage}{0.45\textwidth}
		\centering
		\includegraphics[width=0.3\textwidth]{7-Construccion/Manuales/mobile/tema claro.png}
		\caption{Cambiar tema - Cambio de tema.}
		\label{fig:cambio-tema}
	\end{minipage}
\end{figure}

\subsubsection{Soporte Técnico}
Si encuentras algún problema o tienes alguna pregunta relacionada con el uso de nuestra aplicación, no dudes en contactar a nuestro equipo de soporte técnico.
Puedes encontrar la información de contacto en el apartado “Contacto” de la parte inferior de la página o en el apartado ajustes de la app. Mirar figuras \ref{fig:contacto-movil}, \ref{fig:contacto-app}.
\begin{figure}[H]
	\centering
	\begin{minipage}{0.45\textwidth}
		\centering
		\includegraphics[width=0.3\textwidth]{7-Construccion/Manuales/mobile/contacto.png}
		\caption{Contacto - Información de contacto en la web}
		\label{fig:contacto-movil}
	\end{minipage}
	\hfill
	\begin{minipage}{0.45\textwidth}
		\centering
		\includegraphics[width=0.3\textwidth]{7-Construccion/Manuales/app/soporte.png}
		\caption{Contacto - Información de contacto en la app}
		\label{fig:contacto-app}
	\end{minipage}
\end{figure}

\newpage
\subsubsection{Funciones como Turistas}
\hrulefill

\subsubsection{Reservar una Actividad}
Para reservar una actividad sigue estos pasos:

\begin{enumerate}
	\item Siga las indicaciones de la sección “Exploración de Actividades” para encontrar la actividad deseada y ver la información detallada de la misma.

	\item Selecciona el horario y el idioma deseados dentro de las opciones disponibles y pulsa el botón “Reservar” para empezar la reserva. Mirar figura \ref{fig:disponibilidad-reservar}.
	      Si no has iniciado sesión, se te pedirá que inicies sesión o te registres. Sigue las indicaciones de la sección “Autenticación” para iniciar sesión o registrarte.
	      Se te llevará a una página de confirmación de la reserva, donde verás tus datos personales (se pueden modificar si es necesario), el precio total y un formulario para introducir los datos de pago. Mirar figu \ref{fig:formulario-reservar}.

	\item Completa el formulario con los datos de pago y pulsa el botón “Pagar” para finalizar la reserva.
	      Si la reserva se realiza desde la app móvil, el fomulario de pago se abrirá en un modal al presionar el botón “Pagar” . Mirar figura \ref{fig:formulario-app-reservar}.
	      Se te mostrará una ventana de confirmación de la reserva. Mirar figura \ref{fig:confirmacion-reserva}.

\end{enumerate}

\begin{figure}[H]
	\centering
	\begin{minipage}{0.45\textwidth}
		\centering
		\includegraphics[width=0.8\textwidth]{7-Construccion/Manuales/web/disponibilidad reservar.png}
		\caption{Reservar una actividad \\ Selección de horario e idioma y formulario de reserva.}
		\label{fig:disponibilidad-reservar}
	\end{minipage}
	\hfill
	\begin{minipage}{0.45\textwidth}
		\centering
		\includegraphics[width=0.9\textwidth]{7-Construccion/Manuales/web/formulario-reservar.png}
		\caption{Reservar una actividad \\ Formulario de reserva.}
		\label{fig:formulario-reservar}
	\end{minipage}
\end{figure}

\begin{figure}[H]
	\centering
	\begin{minipage}{0.45\textwidth}
		\centering
		\includegraphics[width=0.35\textwidth]{7-Construccion/Manuales/app/P2-Reservar.png}
		\includegraphics[width=0.35\textwidth]{7-Construccion/Manuales/app/P3-Reservar.png}
		\caption{Reservar una actividad \\ Formulario de reserva en la app.}
		\label{fig:formulario-app-reservar}
	\end{minipage}
	\hfill
	\begin{minipage}{0.45\textwidth}
		\centering
		\includegraphics[width=1\textwidth]{7-Construccion/Manuales/web/thankYou.png}
		\caption{Reservar una actividad \\ Confirmación de la reserva.}
		\label{fig:confirmacion-reserva}
	\end{minipage}
\end{figure}


\subsubsection{Gestión de Reservas}
Para gestionar tus reservas sigue estos pasos:
\begin{enumerate}
	\item Accede al apartado “Reservas” . Mirar figuras \ref{fig:opcion-mobile-reservas}, \ref{fig:opcion-app-reservas}, \ref{fig:opcion-web-reservas}.

	\item Se abrirá una página con un listado de tus reservas activas y pasadas.
	      Para ver más detalles de una reserva, pulsa el botón “Gestionar” de la reserva deseada.
	      Se abrirá una nueva ventana con la información detallada de la reserva, incluyendo la actividad, la fecha y hora, el idioma, el número de personas y el precio total.
	      Mirar figura \ref{fig:detalles-reserva}.

	\item Para cancelar una reserva, pulsa el botón “Cancelar” de la reserva deseada y se abrirá una ventana de confirmación de la cancelación.
	      Toca el botón “Confirmar” para cancelar la reserva. Mirar figura \ref{fig:cancelar-reserva}.

\end{enumerate}

\begin{figure}[H]
	\centering
	\begin{minipage}{0.45\textwidth}
		\centering
		\includegraphics[width=0.3\textwidth]{7-Construccion/Manuales/mobile/menu marcado.png}
		\includegraphics[width=0.3\textwidth]{7-Construccion/Manuales/mobile/reservas marcado.png}
		\caption{Gestión de reservas \\ Despliegue del menú y selección de la opción “Reservas” .}
		\label{fig:opcion-mobile-reservas}
	\end{minipage}
	\hfill
	\begin{minipage}{0.45\textwidth}
		\centering
		\includegraphics[width=0.3\textwidth]{7-Construccion/Manuales/app/P1-GestionReserva.png}
		\caption{Gestión de reservas \\ Selección de la opción “Reservas” de la barra inferior de navegación.}
		\label{fig:opcion-app-reservas}
	\end{minipage}
\end{figure}

\begin{figure}[H]
	\begin{minipage}{0.40\textwidth}
		\centering
		\includegraphics[width=1\textwidth]{7-Construccion/Manuales/web/reservas opcion.png}
		\caption{Gestión de reservas \\ Selección de la opción “Reservas” de la barra superior de navegación.}
		\label{fig:opcion-web-reservas}
	\end{minipage}
	\hfill
	\begin{minipage}{0.5\textwidth}
		\centering
		\includegraphics[width=0.3\textwidth]{7-Construccion/Manuales/mobile/gestionar.png}
		\includegraphics[width=0.65\textwidth]{7-Construccion/Manuales/web/reserva detalles.png}
		\caption{Gestión de reservas \\ Redirección a la información detallada de la reserva.}
		\label{fig:detalles-reserva}
	\end{minipage}
\end{figure}

\begin{figure}[H]
	\centering
	\includegraphics[width=0.2\textwidth]{7-Construccion/Manuales/mobile/cancelar reserva.png}
	\includegraphics[width=0.2\textwidth]{7-Construccion/Manuales/mobile/confirmar cancelacion reserva.png}
	\caption{Gestión de reservas \\ Cancelar reserva.}
	\label{fig:cancelar-reserva}
\end{figure}

\subsubsection{Publicar una Valoración}
Recuerda que puedes publicar una valoración de una actividad después de haberla realizado.

Para publicar una valoración de una actividad desde tu dispositivo móvil, sigue estos pasos:
\begin{enumerate}
	\item Accede al apartado “Reservas” . Mirar figuras \ref{fig:opcion-mobile-reservas}, \ref{fig:opcion-app-reservas}, \ref{fig:opcion-web-reservas}.

	\item Se abrirá una página con un listado de tus reservas activas y pasadas. Toca el botón “Gestionar” de la reserva que deseas valorar.
	      Se abrirá una nueva ventana con la información detallada de la reserva, incluyendo la actividad, la fecha y hora, el idioma, el número de personas y el precio total.
	      Mirar figura \ref{fig:detalles-reserva-completada}.
	\item Para publicar una valoración, pulsa el botón “Añadir valoración” de la reserva deseada. Se abrirá una nueva ventana con un formulario para introducir tu puntuación y comentario (campo no obligatorio).
	      Mirar figura \ref{fig:modal-valoracion}.
	\item Completa el formulario y pulsa el botón “Guardar cambios” para enviar la valoración.
\end{enumerate}

\begin{figure}[H]
	\centering
	\begin{minipage}{0.45\textwidth}
		\centering
		\includegraphics[width=0.3\textwidth]{7-Construccion/Manuales/mobile/gestionar completada.png}
		\includegraphics[width=0.65\textwidth]{7-Construccion/Manuales/web/reserva detalles completada.png}
		\caption{Gestión de reservas \\ Redirección a la información detallada de la reserva.}
		\label{fig:detalles-reserva-completada}
	\end{minipage}
	\hfill
	\begin{minipage}{0.45\textwidth}
		\centering
		\includegraphics[width=0.3\textwidth]{7-Construccion/Manuales/mobile/añadir valoracion.png}
		\includegraphics[width=0.3\textwidth]{7-Construccion/Manuales/mobile/formulario valoracion.png}
		\caption{Gestión de reservas \\ Formulario valoración}
		\label{fig:modal-valoracion}
	\end{minipage}
\end{figure}

\newpage
\subsubsection{Funciones como Administrador}
\hrulefill

\subsubsection{Panel de contol}
Una vez has iniciado sesión, podrás acceder a tu panel de control haciendo clic en la opción “Panel de control” del menú superior de la página de inicio. Mirar figura \ref{fig:panel-control-opcion}.
En tu panel de control, podrás ver estadísticas sobre las reservas realizadas, los beneficios obtenidos, el número de usuarios registrados, reservas recientes y gráficos informativos. Mirar figura \ref{fig:panel-control}.

En esta página también se podrá ir a la sección de usuarios y actividades para poder gestionarlos.

\begin{figure}[H]
	\centering
	\begin{minipage}{0.45\textwidth}
		\centering
		\includegraphics[width=1\textwidth]{7-Construccion/Manuales/web/panel control opcion.png}
		\caption{Dashboard \\ Opción de panel de control}
		\label{fig:panel-control-opcion}
	\end{minipage}
	\hfill
	\begin{minipage}{0.45\textwidth}
		\centering
		\includegraphics[width=1\textwidth]{7-Construccion/Manuales/web/panel control.png}
		\caption{Dashboard \\ Estadísticas del panel de control}
		\label{fig:panel-control}
	\end{minipage}
\end{figure}

\subsubsection{Gestión de usuarios}
Para gestionar los usuarios, una vez has iniciado sesión, sigue estos pasos:
\begin{enumerate}
	\item Haz clic en la opción “Panel de control” del menú superior de la página de inicio. Mirar figura \ref{fig:panel-control-opcion}.
	\item Se abrirá una página con estadísticas y un menú en la parte izquierda. Mirar figura \ref{fig:panel-control}.
	\item Haz clic en la opción “Usuarios” del menú, para acceder a la lista de usuarios registrados. Mirar figura \ref{fig:usuarios-opcion}.
	\item Se pueden filtrar los usuarios por nombre, email o número de identificación, haciendo uso de la barra de búsqueda.
	\item Haz clic en el icono del ojo para acceder a la información detallada del usuario.
	\item Haz clic en el icono del lápiz para editar el usuario.
	\item Haz click en el icono de la papelera para eliminar el usuario de la aplicación.
	\item En la parte superior derecha de la lista de usuarios, se encuentra un botón para añadir un nuevo usuario. Mirar figura \ref{fig:usuarios-add}.
\end{enumerate}

\begin{figure}[H]
	\centering
	\begin{minipage}{0.45\textwidth}
		\centering
		\includegraphics[width=1\textwidth]{7-Construccion/Manuales/web/usuarios opcion.png}
		\caption{Dashboard \\ Opción de usuarios}
		\label{fig:usuarios-opcion}
	\end{minipage}
	\hfill
	\begin{minipage}{0.45\textwidth}
		\centering
		\includegraphics[width=1\textwidth]{7-Construccion/Manuales/web/usuario add.png}
		\caption{Dashboard \\ Añadir usuario}
		\label{fig:usuarios-add}
	\end{minipage}
\end{figure}

\subsubsection{Gestión de actividades}
Para gestionar las actividades, una vez has iniciado sesión, sigue estos pasos:
\begin{enumerate}
	\item Haz clic en la opción “Panel de control” del menú superior de la página de inicio. Mirar figura \ref{fig:panel-control-opcion}.
	\item Se abrirá una página con estadísticas y un menú en la parte izquierda. Mirar figura \ref{fig:panel-control}.
	\item Haz clic en la opción “Actividades” del menú, para acceder a la lista de actividades registradas. Mirar figura \ref{fig:actividades-opcion}.
	\item Se pueden filtrar las actividades por nombre o ubicación, haciendo uso de la barra de búsqueda.
	\item Haz clic en el icono del ojo para acceder a la información detallada de la actividad.
	\item Haz clic en el icono del lápiz para editar la actividad.
	\item Haz click en el icono de la papelera para eliminar la actividad de la aplicación y todos sus eventos asociados. Mirar figura \ref{fig:actividades-ver-eventos}.
	\item En el menú de la parte izquierda, se encuentra:
	      \begin{itemize}
		      \item un botón para activar o desactivar la visibilidad de los eventos de las actividades.
		      \item un rango de fechas para filtrar los eventos por fecha de inicio.
		      \item un campo para mostrar o ocultar los eventos cancelados.
		      \item un botón para añadir un nuevo evento al sistema.
	      \end{itemize}
	\item En la parte superior derecha de la lista de actividades, se encuentra un botón para añadir una nueva actividad. Mirar figura \ref{fig:actividades-add}.
\end{enumerate}

\begin{figure}[H]
	\centering
	\begin{minipage}{0.45\textwidth}
		\centering
		\includegraphics[width=1\textwidth]{7-Construccion/Manuales/web/actividades opcion.png}
		\caption{Dashboard \\ Opción de actividades}
		\label{fig:actividades-opcion}
	\end{minipage}
	\hfill
	\begin{minipage}{0.45\textwidth}
		\centering
		\includegraphics[width=1\textwidth]{7-Construccion/Manuales/web/actividades ver eventos.png}
		\caption{Dashboard \\ Ver eventos}
		\label{fig:actividades-ver-eventos}
	\end{minipage}
\end{figure}

\begin{figure}[H]
	\centering
	\includegraphics[width=0.5\textwidth]{7-Construccion/Manuales/web/actividad add.png}
	\caption{Dashboard \\ Añadir actividad}
	\label{fig:actividades-add}
\end{figure}

\newpage
\subsubsection{Funciones como Guía}
\hrulefill

\subsubsection{Ver eventos proximos}
Para ver los eventos que tendrás próximamente, deberás acceder a “Eventos próximos” . Mirar figuras \ref{fig:eventos-proximos-opcion-movil}, \ref{fig:eventos-proximos-opcion-app}, \ref{fig:eventos-proximos-opcion-web} .
Se abrirá una página donde podrás seleccionar la fecha deseada para ver los eventos programados para ese día.
Y se mostrará una lista con los eventos programados para la fecha seleccionada, incluyendo el nombre, la descripción, la duración, la ubicación, el horario. Mirar figuras \ref{fig:eventos-proximos-lista-web}, \ref{fig:eventos-proximos-lista-mobile}.
Para ver los participantes de un evento, pulsa el botón “Ver participantes” del evento deseado. Mirar figura \ref{fig:eventos-proximos-participantes}.

\begin{figure}[H]
	\centering
	\begin{minipage}{0.45\textwidth}
		\centering
		\includegraphics[width=0.3\textwidth]{7-Construccion/Manuales/mobile/menu marcado.png}
		\includegraphics[width=0.3\textwidth]{7-Construccion/Manuales/mobile/eventos proximos marcado.png}
		\caption{Eventos próximos \\ Despliegue del menú y selección de la opción “Eventos próximos” .}
		\label{fig:eventos-proximos-opcion-movil}
	\end{minipage}
	\hfill
	\begin{minipage}{0.45\textwidth}
		\centering
		\includegraphics[width=0.3\textwidth]{7-Construccion/Manuales/app/eventos marcado.png}
		\caption{Eventos próximos \\ Selección de la opción “Eventos próximos” de la barra inferior de navegación.}
		\label{fig:eventos-proximos-opcion-app}
	\end{minipage}
\end{figure}

\begin{figure}[H]
	\centering
	\includegraphics[width=0.5\textwidth]{7-Construccion/Manuales/web/next events opcion.png}
	\caption{Eventos próximos \\ Selección de la opción “Eventos próximos” de la barra superior de navegación.}
	\label{fig:eventos-proximos-opcion-web}
\end{figure}

\begin{figure}[H]
	\centering
	\begin{minipage}{0.45\textwidth}
		\centering
		\includegraphics[width=1\textwidth]{7-Construccion/Manuales/web/eventos proximos lista.png}
		\caption{Eventos próximos \\ Lista de eventos.}
		\label{fig:eventos-proximos-lista-web}
	\end{minipage}
	\hfill
	\begin{minipage}{0.45\textwidth}
		\centering
		\includegraphics[width=0.3\textwidth]{7-Construccion/Manuales/mobile/eventos proximos.png}
		\includegraphics[width=0.3\textwidth]{7-Construccion/Manuales/mobile/botón calendario eventos.png}
		\includegraphics[width=0.3\textwidth]{7-Construccion/Manuales/mobile/calendario eventos.png}
		\caption{Eventos próximos \\ Calendario de eventos próximos.}
		\label{fig:eventos-proximos-lista-mobile}
	\end{minipage}
\end{figure}

\begin{figure}[H]
	\centering
	\includegraphics[width=0.15\textwidth]{7-Construccion/Manuales/mobile/participantes.png}
	\includegraphics[width=0.15\textwidth]{7-Construccion/Manuales/mobile/lista participantes.png}
	\caption{Eventos próximos \\ Lista de participantes de un evento.}
	\label{fig:eventos-proximos-participantes}
\end{figure}
\subsection{Manual de desarrollador}
Este manual proporciona una guía detallada para preparar el entorno de desarrollo.

\subsubsection{FontEnd}

La aplicación está alojada en un repositorio de Git y se empaquetará utilizando Android Studio y Xcode para plataformas móviles.
Para subirla a la web se utilizará Vercel y solo se necesita llevar los cambios a la rama main para que se despliegue automáticamente.

\subsubsection{Paso 1: Clonar el Repositorio}

\begin{enumerate}
	\item Abre tu terminal o línea de comandos.
	\item Clona el repositorio de Git utilizando el siguiente comando:
	      \begin{lstlisting}[language=bash]
    git clone https://github.com/DelPozoAmoAndres/TurismoApp-Cliente
    \end{lstlisting}
	\item Navega al directorio del proyecto clonado:
	      \begin{lstlisting}[language=bash]
    cd TurismoApp-Cliente
    \end{lstlisting}
\end{enumerate}

\subsubsection{Paso 2: Instalación de Dependencias}

\begin{enumerate}
	\item Asegúrate de estar en el directorio raíz del proyecto.
	\item Instala las dependencias del proyecto utilizando npm:
	      \begin{lstlisting}[language=bash]
    npm install
    \end{lstlisting}
\end{enumerate}

\subsubsection{Paso 3: Configuración de Ionic}

\begin{enumerate}
	\item Instala Ionic CLI globalmente:
	      \begin{lstlisting}[language=bash]
    npm install -g @ionic/cli
    \end{lstlisting}
	\item Verifica la instalación de Ionic:
	      \begin{lstlisting}[language=bash]
    ionic --version
    \end{lstlisting}
\end{enumerate}

\subsubsection{Paso 4: Configuración de Android Studio}

\begin{enumerate}
	\item Abre Android Studio y sigue las instrucciones para instalar las dependencias necesarias, como el SDK de Android.
	\item Configura el entorno de desarrollo para permitir la compilación de aplicaciones Ionic:
	      \begin{itemize}
		      \item Ve a \texttt{File > Settings > Appearance \& Behavior > System Settings > Android SDK}.
		      \item Asegúrate de tener instalados los SDK Platforms y SDK Tools necesarios (Android SDK Build-Tools, Android Emulator, Android SDK Platform-Tools, etc.).
	      \end{itemize}
\end{enumerate}

\subsubsection{Paso 4.1: Configuración de Variables de Entorno}
Para que Android Studio pueda compilar la aplicación, es necesario configurar las variables de entorno necesarias.

En caso de estar utilizando Windows, sigue los siguientes pasos:
\begin {itemize}
\item Abre el Panel de Control y ve a \texttt{Sistema y Seguridad > Sistema > Configuración Avanzada del Sistema}.
\item Haz clic en el botón \texttt{Variables de Entorno}.
\item En la sección de Variables del Sistema, haz clic en \texttt{Nueva} y añade las siguientes variables:
\begin{itemize}
	\item \texttt{ANDROID\_HOME} con el valor de la ruta al directorio de instalación de Android SDK.
	\item \texttt{JAVA\_HOME} con el valor de la ruta al directorio de instalación de Java JDK.
\end{itemize}
\end{itemize}

En caso de estar utilizando macOS, sigue los siguientes pasos:
\begin {itemize}
\item Abre el terminal y edita el archivo \texttt{.bash\_profile} o \texttt{.zshrc}:
\begin{lstlisting}[language=bash]
    nano ~/.bash_profile
    \end{lstlisting}
\item Añade las siguientes líneas al archivo:
\begin{lstlisting}[language=bash]
    export ANDROID_HOME=/Users/USERNAME/Library/Android/sdk
    export JAVA_HOME=/Library/Java/JavaVirtualMachines/jdk1.8.0_291.jdk/Contents/Home
    export PATH=$PATH:$ANDROID_HOME/tools:$ANDROID_HOME/platform-tools
    \end{lstlisting}
\item Guarda los cambios y recarga el archivo de configuración:
\begin{lstlisting}[language=bash]
    source ~/.bash_profile
    \end{lstlisting}
\end {itemize}

\subsubsection{Paso 4.2: Configuración de Dispositivos Virtuales}
Para poder ejecutar la aplicación en un emulador de Android, es necesario configurar un dispositivo virtual en Android Studio. Sigue los siguientes pasos:
\begin {itemize}
\item Abre Android Studio y ve a \texttt{Tools > AVD Manager}.
\item Haz clic en \texttt{Create Virtual Device} y selecciona un dispositivo de la lista.
\item Descarga una imagen de sistema para el dispositivo seleccionado y haz clic en \texttt{Next}.
\item Configura las opciones del dispositivo virtual y haz clic en \texttt{Finish}.
\item Una vez creado el dispositivo virtual, haz clic en el botón de reproducción para iniciar el emulador.
\end {itemize}


\subsubsection{Paso 5: Configuración de Xcode (Solo para macOS)}

\begin{enumerate}
	\item Abre Xcode y asegúrate de tener las herramientas de línea de comandos instaladas:
	      \begin{lstlisting}[language=bash]
    xcode-select --install
    \end{lstlisting}
	\item Asegúrate de tener las últimas versiones de las herramientas necesarias (iOS SDK).
\end{enumerate}

\subsubsection{Paso 6: Ejecución de la Aplicación en un Navegador}

\begin{enumerate}
	\item Para verificar que todo está configurado correctamente, ejecuta la aplicación en tu navegador:
	      \begin{lstlisting}[language=bash]
    ionic serve
    \end{lstlisting}
\end{enumerate}

\subsubsection{Paso 7: Compilación para Android}

\begin{enumerate}
	\item Asegúrate de que tu dispositivo Android esté en modo de desarrollador y conectado a tu computadora, o configura un emulador en Android Studio.
	\item Construye la aplicación para Android:
	      \begin{lstlisting}[language=bash]
    ionic capacitor build android
    \end{lstlisting}
	\item Abre el proyecto en Android Studio:
	      \begin{lstlisting}[language=bash]
    npx cap open android
    \end{lstlisting}
	\item Desde Android Studio, puedes compilar y ejecutar la aplicación en un dispositivo o emulador.
\end{enumerate}

\subsubsection{Paso 8: Compilación para iOS}

\begin{enumerate}
	\item Asegúrate de que tu dispositivo iOS esté conectado a tu Mac, o configura un simulador en Xcode.
	\item Construye la aplicación para iOS:
	      \begin{lstlisting}[language=bash]
    ionic capacitor build ios
    \end{lstlisting}
	\item Abre el proyecto en Xcode:
	      \begin{lstlisting}[language=bash]
    npx cap open ios
    \end{lstlisting}
	\item Desde Xcode, puedes compilar y ejecutar la aplicación en un dispositivo o simulador.
\end{enumerate}

\subsubsection{Backend}

La aplicación está alojada en un repositorio de Git y se desplegará en AWS E2C usando Docker.
El despliegue se realizará utilizando GitHub Actions para automatizar el proceso.

\subsubsection{Paso 1: Clonar el Repositorio}

\begin{enumerate}
	\item Abre tu terminal o línea de comandos.
	\item Clona el repositorio de Git utilizando el siguiente comando:
	      \begin{lstlisting}[language=bash]
    git clone https://github.com/DelPozoAmoAndres/TurismoApp-Server
    \end{lstlisting}
	\item Navega al directorio del proyecto clonado:
	      \begin{lstlisting}[language=bash]
    cd TurismoApp-Server
    \end{lstlisting}
\end{enumerate}

\subsubsection{Paso 2: Instalación de Dependencias}

\begin{enumerate}
	\item Asegúrate de estar en el directorio raíz del proyecto.
	\item Instala las dependencias del proyecto utilizando npm:
	      \begin{lstlisting}[language=bash]
    npm install
    \end{lstlisting}
\end{enumerate}

\subsubsection{Paso 3: Configuración del Entorno}

Crea un archivo de configuración de entorno y renómbralo a \texttt{.env}. Deberás añadir las siguientes variables de entorno:
\begin {itemize}
\item \texttt{CLAVE\_SECRETA\_DE\_STRIPE}: Clave secreta de Stripe.
\item \texttt{JWT\_SECRET}: Clave secreta para firmar los tokens JWT.
\item \texttt{MONGODB\_URI}: URI de conexión a la base de datos MongoDB.
\item \texttt{PORT}: Puerto en el que se ejecutará el servidor.
\end {itemize}

\subsubsection{Paso 4: Ejecución del Servidor Localmente}

\begin{enumerate}
	\item Asegúrate de estar en el directorio raíz del proyecto.
	\item Ejecuta el siguiente comando para iniciar el servidor localmente en modo de desarrollo:
	      \begin{lstlisting}[language=bash]
    npm run dev
    \end{lstlisting}
	\item Para ejecutar el servidor en modo de producción, utiliza los siguientes comandos:
	      \begin{lstlisting}[language=bash]
    npm run build
    npm start
    \end{lstlisting}
	\item El servidor se ejecutará en el puerto especificado en el archivo de configuración del entorno.
	\item Puedes acceder al servidor localmente en tu navegador utilizando la URL \texttt{http://localhost:PUERTO}, donde \texttt{PUERTO} es el número de puerto especificado en el archivo de configuración del entorno.
\end{enumerate}


\subsubsection{Paso 5: Actualizaciones y Despliegues Continuos}

\begin{enumerate}
	\item Cada vez que realices un push a la rama \texttt{main}, GitHub Actions se encargará de construir y desplegar la nueva imagen de Docker en AWS ECS automáticamente. Para ello utilizará las claves secretas configuradas en GitHub y el archivo \texttt{.github/workflows/aws.yml}.
	\item Monitorea el estado del despliegue y asegúrate de que todo funciona correctamente después de cada actualización.
\end{enumerate}




